\documentclass[11pt,oneside]{article}
\usepackage[utf8]{inputenc}
\usepackage[american]{babel}
% \usepackage[
%   paperwidth=6in, paperheight=9in,
%   bindingoffset=0.25in, left=0.75in, right=0.75in, top=0.75in, bottom=1.25in
% ]{geometry}
\usepackage{setspace}
\usepackage{microtype}
\usepackage{mathpazo} % Palantino font
\usepackage{minted}
  \setminted{fontsize=\footnotesize}
  \setminted{breaklines}
  \usemintedstyle{bw}
\usepackage{hyperref}
\pagestyle{empty}
\setstretch{1.2}

\title{Zold, Lightweight Crypto Currency}
\author{Yegor Bugayenko\\\texttt{yegor@zold.io}}

\begin{document}
\raggedbottom
\maketitle
\begin{abstract}
Works for you?
\end{abstract}

\section{Motivation}

Bitcoin, the first decentralized digital currency, was released in
January 2009. Since then a number of similar Blockchain-based products have been
created, including Etherium, Litecoin, and others.

Zold is also a decentralized digital currency that maintains its transactions
in an unpredicable amount of zero-trust server nodes, trying to guarantee
data consistency. However, the architecture of Zold is not based on Blockchain
principles. The development of Zold was motivated by the desire to overcome
a few obvious disadvantages of existing solutions.

First, the speed of transaction processing is rather low.

Second, mining commissions are high.

Third, the technology is too complex.

Zold was created as an attempt to resolve these mentioned problems
of existing digital currencies.

\section{Principles}

\textbf{Open source}.
Zold is a command line tool. Its entire code base is open source.

\textbf{Capacity}.
One currency unit is called ZLD.
One ZLD by convention equals to $2^24$ (16,777,216) \emph{zents}.
All amounts are stored as signed 64-bit integers.
Thus, the technical capacity of the currency is 549,755,813,888 ZLD (half a trillion).

\textbf{Zero wallet}.
There is no mining, the only way to get ZLD is to receive it from someone else.
The wallet with the \texttt{0x00} ID belongs to the
issuer and may have a negative balance. All other wallets
may have only positive balances.

\textbf{No general ledger}.
There is no central ledger, each wallet has its own personal ledger.
Each transaction in the ledger is confirmed by RSA signature;

\textbf{No trust}.
The network of communicating nodes maintains wallets of users.
Anyone can add a node to the network.
It is assumed that any node may contain corrupted data, either by mistake or intentionally.

\section{Proof of Work}

Each node calculates its own score.
First, it builds the initial text body, which consists of four parts,
separated by spaces:

\begin{itemize}
\item The current timestamp in UTC ISO 8601,
\item The host name or IP address, e.g. \texttf{b2.zold.io},
\item The TCP port number,
\item The invoice.
\end{itemize}

For example, the body may look like this:

\begin{minted}{text}
2017-07-19T21:24:51Z b2.zold.io 4096 Yt0lOy6Rgf@0000000000000000
\end{minted}

Then, it attempts to append any
arbitrary text (has to match \texttt{[a-zA-Z0-9]+} regular expression)
to the end of it and to calculate SHA-256 of the text
in the hexadecimal format. For example, this would be the body
with \texttt{abcdef} suffix:

\begin{minted}{text}
2017-07-19T21:24:51Z b2.zold.io 4096 Yt0lOy6Rgf@0000000000000000 abcdef
\end{minted}

The SHA-256 of this body will be:

\begin{minted}{text}
2017-07-19T21:24:51Z b2.zold.io 4096 Yt0lOy6Rgf@0000000000000000 abcdef
\end{minted}

The node attempts to try different sufficies until one of them produces
SHA-256 hash that ends with \texttt{000000} (six zeros). For example, this
suffix \texttt{...} works
(it took about an hour to find it on 2.3GHz Intel Core i7):

When the first suffix is found, the score is 1. Then, to
increase the score by one, the next suffix has to be found, which
can be added to the first 64 characters of the previous hash
in order to obtain a new hash with trailing zeros, for example:

\begin{minted}{text}
2017-07-19T21:24:51Z b2.zold.io 4096 Yt0lOy6Rgf@0000000000000000 abcdef abcdef
\end{minted}

And so on.

The score is valid only when the starting time point is earlier than
current time, but not earlier than 24 hours ago. The strength of the score
is the amount of the trailing zeros in the hash. In the example above the
strength was equal to six.

\section{Wallets}

\section{Push}

\section{Fetch and Merge}

\section{Threats Analysis}

how are you?

\end{document}
